\documentclass[11pt,]{article}
\usepackage[margin=1in]{geometry}
\newcommand*{\authorfont}{\fontfamily{phv}\selectfont}
\usepackage[]{mathpazo}
\usepackage{abstract}
\renewcommand{\abstractname}{}    % clear the title
\renewcommand{\absnamepos}{empty} % originally center
\newcommand{\blankline}{\quad\pagebreak[2]}

\providecommand{\tightlist}{%
  \setlength{\itemsep}{0pt}\setlength{\parskip}{0pt}} 
\usepackage{longtable,booktabs}

\usepackage{parskip}
\usepackage{titlesec}
\titlespacing\section{0pt}{12pt plus 4pt minus 2pt}{6pt plus 2pt minus 2pt}
\titlespacing\subsection{0pt}{12pt plus 4pt minus 2pt}{6pt plus 2pt minus 2pt}

\titleformat*{\subsubsection}{\normalsize\itshape}

\usepackage{titling}
\setlength{\droptitle}{-.25cm}

%\setlength{\parindent}{0pt}
%\setlength{\parskip}{6pt plus 2pt minus 1pt}
%\setlength{\emergencystretch}{3em}  % prevent overfull lines 

\usepackage[T1]{fontenc}
\usepackage[utf8]{inputenc}

\usepackage{fancyhdr}
\pagestyle{fancy}
\usepackage{lastpage}
\renewcommand{\headrulewidth}{0.3pt}
\renewcommand{\footrulewidth}{0.0pt} 
\lhead{}
\chead{}
\rhead{\footnotesize Data Analysis for Policy Research Using R -- Fall
2020}
\lfoot{}
\cfoot{\small \thepage/\pageref*{LastPage}}
\rfoot{}

\fancypagestyle{firststyle}
{
\renewcommand{\headrulewidth}{0pt}%
   \fancyhf{}
   \fancyfoot[C]{\small \thepage/\pageref*{LastPage}}
}

%\def\labelitemi{--}
%\usepackage{enumitem}
%\setitemize[0]{leftmargin=25pt}
%\setenumerate[0]{leftmargin=25pt}




\makeatletter
\@ifpackageloaded{hyperref}{}{%
\ifxetex
  \usepackage[setpagesize=false, % page size defined by xetex
              unicode=false, % unicode breaks when used with xetex
              xetex]{hyperref}
\else
  \usepackage[unicode=true]{hyperref}
\fi
}
\@ifpackageloaded{color}{
    \PassOptionsToPackage{usenames,dvipsnames}{color}
}{%
    \usepackage[usenames,dvipsnames]{color}
}
\makeatother
\hypersetup{breaklinks=true,
            bookmarks=true,
            pdfauthor={ ()},
             pdfkeywords = {},  
            pdftitle={Data Analysis for Policy Research Using R},
            colorlinks=true,
            citecolor=blue,
            urlcolor=blue,
            linkcolor=magenta,
            pdfborder={0 0 0}}
\urlstyle{same}  % don't use monospace font for urls


\setcounter{secnumdepth}{0}





\usepackage{setspace}

\title{Data Analysis for Policy Research Using R}
\author{Columbia \textbar{} SIPA}
\date{Fall 2020}


\begin{document}  

		\maketitle
		
	
		\thispagestyle{firststyle}

%	\thispagestyle{empty}


	\noindent \begin{tabular*}{\textwidth}{ @{\extracolsep{\fill}} lr @{\extracolsep{\fill}}}

Instructor: Harold Stolper & Class: Tues 2:10-4pm, online\\
Pronouns: he/they & Recitation: Thurs 6:10-8pm, online\\
E-mail: \texttt{\href{mailto:hbs2103@columbia.edu}{\nolinkurl{hbs2103@columbia.edu}}} & TA: Kevin
Wibisono (kw2870)\\
OH: Walk-in (Zoom) Tues 10:30-11:30am \textbar{}
\href{https://helloharold.youcanbook.me}{book} indiv.
appt.  &  TA OH: Mon 11:30am-1:30pm\\
Course website: \texttt{\url{https://hreplots.github.io/U6614/}} & (sign up via Google Sheets)\\
	&  \\
	\hline
	\end{tabular*}
	
\vspace{2mm}
	


\hypertarget{course-description}{%
\section{Course Description}\label{course-description}}

This course will develop the skills to prepare, analyze, and present
data for policy analysis and program evaluation using R. In Quant I and
II, students are introduced to probability and statistics, regression
analysis and causal inference. In this course we focus on the practical
application of these skills to explore data and policy questions on your
own. The goal is to help students become effective analysts and policy
researchers: given available data, what sort of analysis would best
inform our policy questions? How do we prepare data and implement
statistical methods using R? How can we begin to draw conclusions about
the causal effects of policies, not just correlation?

We'll learn these skills by exploring data on a range of policy topics:
COVID-19 cases; racial bias in NYPD subway fare evasion enforcement; the
distribution of Village Fund grants in Indonesia; US police shootings;
wage gaps by gender/race; and student projects on topics of your
choosing.

\hypertarget{course-learning-goals}{%
\section{Course Learning Goals}\label{course-learning-goals}}

We will focus on developing skills in the following areas:

\begin{itemize}
\item
  \textbf{Research design:} understanding how data structure impacts
  analysis and causal inference
\item
  \textbf{Data management:} cleaning and structuring data for analysis
\item
  \textbf{Exploratory analysis:} identifying and analyzing key factors
  in your analysis
\item
  \textbf{Explanatory analysis:} estimating relationships between
  variables to inform policy
\item
  \textbf{Data visualization and presentation:} conveying findings to
  your target audience
\item
  \textbf{Policy writing and interpretation:} translating statistical
  analysis in accessible terms
\item
  \textbf{R programming skills} (these skills support all above the
  areas)
\end{itemize}

\hypertarget{prerequisite-requirements}{%
\section{Prerequisite Requirements}\label{prerequisite-requirements}}

\begin{enumerate}
\def\labelenumi{\arabic{enumi}.}
\item
  Students should have some very basic exposure to R, or a demonstrated
  aptitude for object-oriented programming languages.
\item
  Students should have completed both U6500 and U6501 (Quant I and II)
  or equivalent.
\end{enumerate}

\hypertarget{required-software}{%
\section{Required Software}\label{required-software}}

The course will be taught using R, a free, open-source programming
language. R has become the most popular language for statistical
analysis in many circles. One advantage to using R is the thousands of
open-source ``libraries'' created by R users. By learning R you'll be
able to carry out practically any statistical method and access powerful
capabilities for data collection, manipulation, and visualization. It is
necessarily more complex than Stata, but far more flexible.

We'll be working with R using R Studio. Instructions on installing R and
R studio can be found at \url{https://stat545.com/install.html}. Please
install both R and R studio on your laptop prior to our first class
session.

\hypertarget{course-structure-and-approach}{%
\section{Course Structure and
Approach}\label{course-structure-and-approach}}

\hypertarget{course-structure}{%
\subsection{Course Structure}\label{course-structure}}

This course will primarily consist of:

\begin{enumerate}
\def\labelenumi{\arabic{enumi}.}
\item
  \textbf{Asynchronous (pre-class) lessons} will be shared via the
  course website with the expectation that students work through them
  independently in advance of class. The idea is to introduce key
  concepts and syntax in R, as well as methodological issues, to prepare
  for in-class discussion and data exercises. This asynchronous content
  will often take the form of web-based lessons (html files) including
  sample code and output that students can try out on their own as they
  go. In some weeks asynchronous materials will also focus on
  policy-relevant examples of data analysis tools and challenges, and
  may include short readings. Each class will begin with a short Zoom
  quiz on this asynchronous content to encourage engagement in advance
  of class. This will be followed by some in-class time for discussion
  to engage with the material, often using Zoom breakout rooms and Poll
  Everywhere. Pre-class lecture content for Tuesday's class session will
  be posted by the previous Thursday.
\item
  \textbf{In-class workshop-style instruction using R} will take up the
  majority of our in-class time together most weeks. We'll be working
  through R code together using R Studio to prepare and explore data for
  analysis.
\item
  \textbf{Four weekly data assignments and short write-ups (``data
  memos'')} which will require you to expand on the work we do together
  in class and write up your work using clear, accessible language. We
  will introduce R Markdown as a tool for you to write up your work and
  present code and findings in a single document. Data memos will be due
  before midnight on Mondays, in advance of Tuesday's class session.
\item
  \textbf{A data project} of students' choosing (with instructor
  approval) to be conducted in consultation with the teaching team and
  presented and submitted towards the end of the semester. Students are
  required to work in \emph{\textbf{groups of two}}. The project will
  require you to use R to explore a policy-relevant research question
  with readily available data. It must focus on analyzing the effect of
  at least one independent variable of interest on some relevant outcome
  variable, though the majority of work you do will involve data
  cleaning, manipulation, and exploratory data analysis to inform the
  specification of appropriate statistical models. In the latter half of
  the class, student groups are \emph{\textbf{required}} to sign-up for
  three individual meetings with the instructor and TA to discuss
  project progress.
\item
  \textbf{A course discussion board} where students can ask homework
  questions/comments to share with classmates and the teaching team. If
  you're stuck or experiencing problems with R more generally, odds are
  others are too. Posting questions and concerns allows us all to
  benefit from each others knowledge. We'll be using \textbf{Piazza} via
  Courseworks for our online discussion. When asking questions on
  Piazza, please include as many details to replicate the ``error'' (if
  applicable), insert code, screenshots, and text to your posts. The
  teaching team will do our best to reply within 48-72 hours, but you
  are all encouraged to share your thoughts/answers on posts by your
  classmates. Writing out explanations to student questions will improve
  your own knowledge and benefit your classmates. Thoughtful
  contributions will also count towards your overall class participation
  grade.
\item
  \textbf{Recitation and office hours.} During recitation time, the TA
  will review student questions about the material introduced that week
  and hold group office hours. The TA will also hold individual office
  hours each week, with a particular focus on assisting students with
  their projects in the second half of the semester.
\end{enumerate}

The instructor will hold both group/walk-in office hours, and individual
office hours by appointment.

\hypertarget{approach-to-learning-r}{%
\subsection{Approach to Learning R}\label{approach-to-learning-r}}

Our approach will emphasize ``learning by doing'' by working through R
code together in class to explore data. Lecture content will introduce
key concepts in advance of class workshop time, to prepare us for the
workshop exercise. Assignments will task you with refining and expanding
the code from in-class workshop exercises, putting your new knowledge to
work.

It will take us some time to build up the skills to effectively explore
messy, real-world data. Learning a new programming language can be
overwhelming, and this class is only the beginning. The goal of this
course is not to become proficient in the sense of memorizing all the
commands you think you will need, but rather to understand the basics of
R syntax and develop the comfort level to explore new functionality and
troubleshoot on your own.

Online resources and coding ``cheat sheets'' will be shared each week,
but learning how to find and employ answers from both within R Studio
and using Google will be among your most valuable resources.

When applicable, chapters from the following open-source resources will
be listed as supplementary learning resources:

\begin{itemize}
\tightlist
\item
  Bryan, J. (2018). \emph{STAT 545: Data wrangling, exploration, and
  analysis with R}. Retrieved from \url{https://stat545.com}.
\item
  Grolemund, G., \& Wickham, H. (2018). \emph{R for Data Science}.
  Retrieved from \url{http://r4ds.had.co.nz}.
\item
  Xie, Y., Allaire, J. j., \& Grolemund, G. (2018). \emph{R Markdown:
  The Definitive Guide}. Retrieved from
  \url{https://bookdown.org/yihui/rmarkdown}.
\end{itemize}

\hypertarget{data-community}{%
\subsection{Data Community}\label{data-community}}

In-class exercises and discussion are designed to foster a data
community where students can interact among themselves and with the
teaching team to share ideas. Data and coding obstacles generally feel
less overwhelming when you can exchange ideas with others. The
Courseworks discussion board will also help us collectively interact
around data and coding issues and learn from each other.

\hypertarget{assignments-grading-and-course-requirements}{%
\section{Assignments, Grading and Course
Requirements}\label{assignments-grading-and-course-requirements}}

\hypertarget{four-weekly-assignments-data-memos-30-of-your-overall-grade-4-x-7.5}{%
\subsection{Four Weekly Assignments (Data Memos) (30\% of your overall
grade -- 4 x
7.5)}\label{four-weekly-assignments-data-memos-30-of-your-overall-grade-4-x-7.5}}

Weekly assignments are due by midnight on Monday night. Assignments will
be graded on a check plus/minus scale. Late submissions will not receive
a grade as we will be discussing solutions during class.

\hypertarget{individual-student-projects-and-required-meetings-50}{%
\subsection{Individual Student Projects and Required Meetings
(50\%)}\label{individual-student-projects-and-required-meetings-50}}

Your project grade will include an-class presentation of your work
to-date near the end of the semester (20\% of your total grade), and a
short report (30\% of your total grade). The data project will also
involve three required meetings with the teaching team for project
advising, and include several intermediate deliverables: (1) submitting
research ideas; and (2) and a proposal with summary statistics.
Intermediate deliverables will not receive their own grade, but late
intermediate submissions will result in a one grade deduction from your
overall project grade for every day late (e.g.~from an A to A-).

\hypertarget{zoom-quizzes-on-asynchronous-pre-lecture-material-10}{%
\subsection{Zoom Quizzes on Asynchronous (Pre-Lecture) Material
(10\%)}\label{zoom-quizzes-on-asynchronous-pre-lecture-material-10}}

Zoom quizzes at the beginning of class will account for 10\% of your
total grade. These quizzes will consist of multiple choice questions,
and are designed to encourage you to engage with the asynchronous
(pre-class) lessons in advance of class so we can put new R
functionality to work in class, and focus on application and discussion.

\hypertarget{in-class-participation-10}{%
\subsection{In-class Participation
(10\%)}\label{in-class-participation-10}}

Students are required to attend weekly class sessions, with their
webcams turned on. You're expected to participate in the weekly class
sessions and discussion, share responses to Poll Everywhere questions
when appropriate, and participate in the Piazza discussion board in
Courseworks. This component can make the difference between an A and B
for your overall course grade, for example, so please come to class
prepared and ready to participate.

\hypertarget{attendance}{%
\subsection{Attendance}\label{attendance}}

Attending synchronous class sessions in Zoom is required, \textbf{with
your webcam turned on}. Multiple unexcused absences may result in
additional deductions to your overall course grade beyond any deductions
for forgone participation.

\hypertarget{course-policies}{%
\section{Course Policies}\label{course-policies}}

\hypertarget{virtual-classroom-environment}{%
\subsection{Virtual Classroom
Environment}\label{virtual-classroom-environment}}

SIPA's greatest asset is the diversity of students, but it also means
being mindful that what we say affects others in ways we may not fully
understand.

Learning R and trying to get a handle on unfamiliar data can feel
overwhelming at times. It's important that we all help create an
environment where students feel comfortable asking questions and talking
about what they don't understand.

After registration closes, community guidelines for Zoom participation
will be set with student input.

\hypertarget{towards-an-anti-racist-learning-experience}{%
\subsection{Towards an Anti-Racist Learning
Experience}\label{towards-an-anti-racist-learning-experience}}

Every class should be an anti-racist class, even when the subject matter
is broadly oriented. In this class we'll cover examples that reflect
systemic gaps based on race, ethnicity, immigration status, and gender
identity, among other aspects of personal identity. Given our focus on
statistical methods, we are limited in the time we can spend discussing
all of the policy context contributing to these gaps (if there is
interest, we can make time for more discussion!). But it is critical to
acknowledge that the social and economic marginalization reflected in
the data is rooted in systemic oppression that upholds opportunity for
some at the expense of others. We should all be thinking about our own
role in upholding these systems.

\hypertarget{teaching-team-communication-and-student-support}{%
\subsection{Teaching Team Communication and Student
Support}\label{teaching-team-communication-and-student-support}}

Given the large of number of student inquiries over a virtual
environment, we ask that you rely on scheduled office hours and the
Courseworks discussion board as much as possible. The instructor will
hold group office hours that are open to all without an appointment, as
well as individual appointment slots that you can book in advance at
\url{https://helloharold.youcanbook.me}. We'll do our best as a teaching
team to respond to inquiries within 72 hours.

While late submissions will not be accepted out of fairness, we
understand many of us are dealing with a great deal of stress and
uncertainty right now. If you are experiencing unexpected challenges
that are affecting your ability to meet your course obligations, I
encourage you to reach out to the teaching team in advance of any
looming deadlines.

\hypertarget{academic-integrity}{%
\subsection{Academic Integrity}\label{academic-integrity}}

SIPA does not tolerate cheating or plagiarism in any form. Students who
violate the Code of Academic \& Professional Conduct will be subject to
the Dean's Disciplinary Procedures. Please consult the code of conduct
\href{http://bulletin.columbia.edu/sipa/academic-policies/discipline-procedures/index.html}{here}.

While grading your assignments, if we come across answers to parts of
any assignments that are clearly not your own words, all involved
parties will receive a zero for those parts and may be referred to
Academic Affairs if appropriate.

\hypertarget{disability-accomodations}{%
\subsection{Disability Accomodations}\label{disability-accomodations}}

SIPA is committed to ensuring that students registered with Columbia
University's Disability Services (DS) receive the reasonable
accommodations necessary to fully participate in their academic
programs. The teaching team will work with SIPA's DS liaison to make
sure the necessary accommodations are provided. You are encouraged to
make an appointment with the instructor to discuss any concerns you have
about your accommodations.

\hypertarget{course-schedule}{%
\section{Course Schedule}\label{course-schedule}}

The syllabus is subject to change at the discretion of the instructor
with proper notice to the students. Students are likely to have varying
levels of statistical knowledge and experience with R. Because it is
difficult to anticipate the optimal pace for students in this class, the
following schedule should be treated as a guide. Topics may carry-over
into the following week(s), and we may end up cutting/adding/re-ordering
later topics based on student needs and interest.

\hypertarget{week-1-982020-introduction-r-basics-and-workflow}{%
\subsection{Week 1, 9/8/2020: Introduction, R Basics and
Workflow}\label{week-1-982020-introduction-r-basics-and-workflow}}

\begin{itemize}
\tightlist
\item
  In-class data: gapminder
\item
  \emph{Assignment 1 posted after class: due by midnight on Monday,
  9/14/2020} \medskip
\end{itemize}

\hypertarget{week-2-9152020-data-types-structures-r-markdown-intro-to-the-tidyverse}{%
\subsection{Week 2, 9/15/2020: Data Types \& Structures, R Markdown,
Intro to the
Tidyverse}\label{week-2-9152020-data-types-structures-r-markdown-intro-to-the-tidyverse}}

\begin{itemize}
\tightlist
\item
  In-class data: country-level COVID-19 case data
\item
  \emph{Assignment 2 posted after class: due by midnight on Monday,
  9/21/2020} \medskip
\end{itemize}

\hypertarget{week-3-9222020-importing-cleaning-summarizing-data-intro-to-ggplot}{%
\subsection{Week 3, 9/22/2020: Importing, Cleaning \& Summarizing Data,
Intro to
ggplot}\label{week-3-9222020-importing-cleaning-summarizing-data-intro-to-ggplot}}

\begin{itemize}
\tightlist
\item
  In-class data: Brooklyn subway fare evasion arrest data part 1:
  cleaning microdata
\item
  \emph{Assignment 3 posted after class: due by midnight on Monday,
  9/28/2020} \medskip
\end{itemize}

\hypertarget{week-4-9292020-joins-aggregation-inference-regression}{%
\subsection{Week 4, 9/29/2020: Joins, Aggregation, Inference \&
Regression}\label{week-4-9292020-joins-aggregation-inference-regression}}

\begin{itemize}
\tightlist
\item
  In-class data: Brooklyn subway fare evasion arrest data part 2:
  aggregation to subway station-level observations
\item
  \emph{Assignment 4 posted after class: due by midnight on Monday,
  10/12/2020} \medskip
\end{itemize}

\hypertarget{week-5-1062020-research-design-regression-analysis-weighting}{%
\subsection{Week 5, 10/6/2020: Research Design, Regression Analysis,
Weighting}\label{week-5-1062020-research-design-regression-analysis-weighting}}

\begin{itemize}
\tightlist
\item
  In-class data: Brooklyn subway fare evasion arrest data part 3:
  poverty, race and enforcement
\item
  \emph{Assignment 4 due before class (by midnight on Monday,
  10/12/2020)} \medskip
\end{itemize}

\hypertarget{week-6-10132020-data-visualization-with-ggplot}{%
\subsection{Week 6, 10/13/2020: Data Visualization with
ggplot}\label{week-6-10132020-data-visualization-with-ggplot}}

\begin{itemize}
\tightlist
\item
  \emph{Assignment: find a partner for your data project.} \medskip
\end{itemize}

\hypertarget{week-7-10202020-working-with-panel-data-part-1}{%
\subsection{Week 7, 10/20/2020: Working with Panel Data (Part
1)}\label{week-7-10202020-working-with-panel-data-part-1}}

\begin{itemize}
\tightlist
\item
  In-class data: Detroit water shutoffs, public health and disparate
  impact by race/income
\item
  \emph{Project deliverable \#1: Submit 2 possible research questions by
  Friday, 10/23 at 5pm.} \medskip
\end{itemize}

\hypertarget{week-8-10272020-working-with-panel-data-part-2}{%
\subsection{Week 8, 10/27/2020: Working with Panel Data (Part
2)}\label{week-8-10272020-working-with-panel-data-part-2}}

\begin{itemize}
\tightlist
\item
  In-class data: Detroit water shutoffs, public health and disparate
  impact by race/income
\item
  \emph{Required meeting \#1 with instructor} \medskip
\end{itemize}

\hypertarget{week-9-1132020-no-class---election-day}{%
\subsection{Week 9, 11/3/2020: NO CLASS - ELECTION
DAY}\label{week-9-1132020-no-class---election-day}}

\begin{itemize}
\tightlist
\item
  \emph{Project deliverable \#2: Mini-proposal with summary statistics
  due before 11:59pm on Friday, 11/6.} \medskip
\end{itemize}

\hypertarget{week-10-11102020-working-with-string-date-variables}{%
\subsection{Week 10, 11/10/2020: Working with String \& Date
Variables}\label{week-10-11102020-working-with-string-date-variables}}

\begin{itemize}
\tightlist
\item
  In-class data: Indonesian Village Fund data
\item
  \emph{Required meeting \#2 with Instructor}
\end{itemize}

\medskip

\hypertarget{week-11-11172020-mapping-with-r}{%
\subsection{Week 11, 11/17/2020: Mapping with
R}\label{week-11-11172020-mapping-with-r}}

\begin{itemize}
\tightlist
\item
  In-class data: Indonesian Village Fund data
\item
  \emph{Required meeting \#3 with TA} \medskip
\end{itemize}

\hypertarget{week-12-11242020-data-visualization-principles-data-project-workshop}{%
\subsection{Week 12, 11/24/2020: Data Visualization Principles, Data
Project
Workshop}\label{week-12-11242020-data-visualization-principles-data-project-workshop}}

\medskip

\hypertarget{week-13-12012020-final-presentations}{%
\subsection{Week 13, 12/01/2020: Final
Presentations}\label{week-13-12012020-final-presentations}}

\medskip

\hypertarget{week-14-12082020-final-presentations}{%
\subsection{Week 14, 12/08/2020: Final
Presentations}\label{week-14-12082020-final-presentations}}

\medskip

\hypertarget{final-papers-due-12182020}{%
\subsection{Final papers due
12/18/2020}\label{final-papers-due-12182020}}




\end{document}

\makeatletter
\def\@maketitle{%
  \newpage
%  \null
%  \vskip 2em%
%  \begin{center}%
  \let \footnote \thanks
    {\fontsize{18}{20}\selectfont\raggedright  \setlength{\parindent}{0pt} \@title \par}%
}
%\fi
\makeatother
